\section*{Introduction}
Although maize is naturally an outcrossing organism, modern breeding takes advantage of highly inbred lines that are used in controlled crosses to produce hybrids. 
Hybrid maize, first developed in the early 20th century by \jri{East and Shull?}, was so successful that within a few decades it had completely replaced mass-selected open pollinated varieties in the United States \citep{crabb1947hybrid}. 
The shift towards selection of inbred lines based on their ability to generate good hybrids – referred to as ‘combining ability’ – constituted an abrupt change from the open-pollinated mass selection that breeders practiced for millennia \citep{anderson1944sources, troyer1999background}.
Maize inbred lines are now partitioned into separate ‘heterotic groups’ that maximize performance and hybrid vigor (heterosis) when inbreds from different heterotic groups are crossed with each other \citep{tracy2006historical}. 

Multiple studies with molecular markers have indicated that these heterotic groups have diverged genetically over time to become highly structured and isolated populations, resulting in a dramatic restructuring of population genetic variation \citep{duvick2004long, ho2005extent, feng2006temporal}. 
Advances in high throughput genotyping and the development of a maize reference genome now enable the observation of maize population structure at high marker density across the whole genome \citep{ganal2011a-large,chia2012maize}. 
These studies have examined a broad spectrum of germplasm at various points in the history of maize to search for the signals of population structure and artificial selection \citep{Hufford2012b, van2012historical}. \jri{add more}
Although selective sweeps remaining from domestication are clearly visible, the impact of selection during modern breeding appears comparatively small in terms of its impact on genomic diversity \citep{Hufford2012b,van2012historical}, despite steady, heritable improvement in phenotype \citep{duvick2005contribution}. 
The lack of distinct selection signals from modern breeding may be due to specific selective events occurring in different populations, necessitating a more focused look within heterotic groups or even single breeding programs. 
\jri{add text about golden glow, krug, and lamkey}

Here we take a detailed look at the dynamics of genome-wide patterns of diversity over time within an individual breeding program. 
The Iowa Stiff Stalk Synthetic (BSSS) and the Iowa Corn Borer Synthetic No. 1 (BSCB1) Reciprocal Recurrent Selection Program of the USDA-ARS at Ames, Iowa (hereafter referred to as the Iowa RRS) represents one of the best-documented public experiments on selection for combining ability and hybrid performance. 
BSSS and BSCB1 have been recurrently selected for improved cross-population hybrids \citep{penny1971twenty}. 
\jri{add some text explaining recurrent selection}
This model of selection, named reciprocal recurrent selection, provides the generalized model for strategies commonly used in commercial maize breeding \citep{comstock1949breeding, duvick2004long}. 
The Iowa RRS experiment proves especially relevant because lines derived from the BSSS population have had a major impact upon the development of commercial hybrids \citep{duvick2004long, darrah19861985}, the formation of modern heterotic groups \citep{troyer1999background; senior1998utility}, and the choice of a maize reference genome \citep{schnable2009the-b73-maize}.
\jri{something about heterosis}