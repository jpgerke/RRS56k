\section*{Discussion}
 
\jri{add in \citep{lorenz2015selection} also include \citep{lamkey2014relative}}

In the Iowa RRS experiment, the BSSS and BSCB1 populations steadily lost genetic diversity as they became more differentiated from one another, and principal component analysis shows that as the effective population size and the rates of inbreeding were altered, the rates of change in population structure were altered as well. 
These patterns of population structure, diversity, and differentiation between BSSS and BSCB1 can be largely reproduced by simulation without any selection, supporting the hypothesis that the majority of the genetic structure observed can be attributed to genetic drift alone, despite effective selection for phenotypic improvement. 
This drift was driven by inbreeding through self-pollination and the small effective population sizes used to select a limited number of high-performing, potentially related individuals at each cycle.  

Reciprocal recurrent selection serves as a model for the method of hybrid maize improvement employed broadly in North America \citep{duvick2004long}.
The trends observed in the Iowa RRS are also a common theme in similar recurrent selection experiments \citep{Romay2013,lamkey2014relative}. 
Increased population structure due to increased inbreeding and lowered effective population size are visible in the modern ‘stiff-stalk’ and ‘non stiff-stalk’ heterotic groups as well \citep{van2012historical}. 
Given the lack of strong selection signals in each case, genetic drift has most likely played a large role in the current genetic structure of modern maize. 
	
Several key inbreds in the ‘stiff-stalk’ heterotic group, B73, B37 and B14, were derived from the BSSS population \citep{darrah19861985, 1986; troyer1999background}. 
B37 and B14 were derived from cycle 0, and B73 was derived from a half-sib recurrent selection program also started with the BSSS population. 
We examined these three inbreds at the pericentromeric regions listed in Table \ref{tabfix} \jri{correct?}, and found that in most cases they carry different haplotypes from those that rose to high frequency in the RRS experiment.  
Near isogenic lines that substitute these centromeric haplotypes from key inbreds could be used to determine if the haplotypes at high frequency in cycle 16 of BSSS provide a phenotypic advantage when crossed with BSCB1. 
Both our study and previous work \citep{labate1997molecular} identified genetic differentiation between cycle 0 of BSSS and the founder inbreds. 
The loss of diversity in the cycle 0 population likely led to the loss of some rare alleles and haplotypes. 
The BSSS was maintained for several years before RRS was initiated, and the differentiation could have occurred due to low effective population size during maintenance. 
If so, then this drift will have impacted the trajectory of the RRS program and of the key inbreds derived from BSSS. 
However, cycle 0 has also undergone maintenance since the beginning of the RRS experiment, so some genetic drift may have occurred during this time as well.
	
Although drift can explain most of the genetic structure genome-wide, phenotypic data provide clear evidence that selection has altered the frequencies of favorable alleles in the BSSS and BSCB1 populations. 
Numerous experiments have shown that the selected populations and the hybrids formed from them are superior to their cycle 0 ancestors \citep{smith1983evaluation, keeratinijakal1993responses, schnicker1993interpopulation, holthaus1995population}. 
Improvement occurred not only for hybrid yield, but concurrently for plant architecture and tolerance to high-density planting in both hybrids and inbreds \citep{brekke2011selection, brekke2011selectionb, edwards2011changes, lauer2012morphological}. 
Since these same trends are observed across wider North American germplasm \citep{tollenaar1989genetic}, identification and characterization of the loci conferring phenotypic improvement in BSSS and BSCB1 could reveal some of the genetic mechanisms by which maize yield has steadily improved over the decades. 
We find that the genomic regions of extremely low diversity evident at cycle 16 are unlikely to be produced by simulation, and heterozygosity falls more than expected across the genome as a whole.  
However, simulated and observed values are often close, so overall the signature of selection is difficult to detect at any given locus. 
This is due in large part to a limitation of statistical power imposed by inbreeding and low effective population sizes, though uncertainty about the fine-scale recombination map also plays a role. 
We show that an identity-by-descent, haplotype based approach provides additional power as it can distinguish between the fixation of rare and common haplotypes. 
However, the results of single locus simulations are sensitive to sampling error and drift caused by population maintenance. 
In this case, it is difficult to assess significance with only a single population to test. This type of analysis is more effective across several replicated populations, which can control for genetic drift due to the independence between the selections in each replicate \citep{lamkey2014relative}. 
	
\citet{van2012historical} assessed variation in and tested for selection across a wider range of North American maize germplasm using the same SNP array. 
Although selected loci were identified in that study, the overall effects on genomic diversity and haplotype patterns were relatively small compared with the impact of domestication. 
That study was a wide survey across many germplasm sources. 
The limited impact of selection could arise because different haplotypes and loci are selected in different breeding programs. 
In support of this idea, we find that the most likely targets of selection between the BSSS and BSCB1 populations occur at non-overlapping loci. 
This non-overlap between putatively selected loci in complementary populations has also been observed in commercial breeding programs \citep{feng2006temporal}.

The observation that the same targets of selection are not observed in the opposing heterotic populations bears implications for the genetic mechanisms responsible for heterosis and the success of maize hybrids. 
Classic overdominance models of heterosis predict that at a single locus, two distinct alleles confer heterozygote advantage when combined, while the dominance model predicts that heterosis is driven by dominance effects and the complementation of linked alleles in low-recombination regions (dominance or pseudo-overdominance). 
In the case of true overdominance, we expect selection should lead to decreased heterozygosity at a locus in both populations as complementary haplotypes are fixed in each group. 
We find no evidence of this genetic phenomenon. 
The observed pattern instead favors a dominance model, where fixation of a haplotype in one population simply selects against that same haplotype in the other population. 
Because most deleterious alleles are rare in both heterotic groups \citep{Mezmouk2014}, most haplotypes in the second population will have a different suite of deleterious variants and will complement the fixed haplotype reasonably well, such that selection will have little impact on diversity in the second population. 
Our data are consistent with such a process having been important for hybrid improvement in the Iowa RSS experiment, and the lack of extreme differentiation seen over time across US Corn Belt germplasm is consistent with a role for such selection in maize hybrid improvement in general \citep{van2012historical}.

Our results are consistent with the dominance and pseudo-overdominance models, but we caution that the exact outcome of particular models will depend strongly on the effects of selection, drift, and the frequencies of beneficial alleles. 
Disentangling these factors remains a major challenge that will require careful simulation of maize breeding using a wider range of parameters. 
This is especially true because in a model of hybrid complementation, genetic drift in one population can alter the selective value of alleles in the other population. 
When drift and selection interact in such a manner, tests for selection that attempt to independently partition the effects of each force may not provide the full picture. Furthermore, uncertainties in the fine-scale relationship between genetic and physical distance make it difficult to assign significance in forward-simulation approaches. 
In the end, the best test for selection of specific genomic regions will ultimately be conducted by phenotypic observation in the field in balanced tests of different haplotypes. 
In the case of individual breeding programs, genome-wide genotyping such as we conducted here can identify the lines carrying the recombinant haplotypes and introgressions necessary to conduct such an experiment. 
